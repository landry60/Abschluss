% deckblatt.tex, 2019/03/15
\documentclass[a4paper,12pt]{report}
%Inhaltsverzeichnis in Kapitel und Sektionen aufgeteilt  \chapter{title} > \section{title} > subsection{title} \subsubsection{title}

% Pakete und Paket-Configs
\usepackage{times} % Times Roman als Standardschrift
\usepackage[ngerman]{babel} % neue deutsche Rechtschreibung und Trennung
\usepackage{fancyhdr} % spezielle Kopfzeilen
\usepackage[utf8]{inputenc} % Umlaute ��� auch normal benutzen und nicht maskieren
\usepackage[babel, german=quotes]{csquotes}
\usepackage{subfigure} % Figures divided into subfigures.
\usepackage{ifthen} % Ermöglicht ifthenelse und whiledo
\usepackage{amsfonts} % Extra mathematical symbols
\usepackage[rflt]{floatflt} % verbessertes floatfig, als um figure's fliesende texte
\usepackage[T1]{fontenc} % ?? aber notwendig für korrekte PDF-Metadaten
%\usepackage{longtable} % Support for tables longer than a page.
%\usepackage{a4wide} % Increases width of printed area of an a4 page.
%\usepackage{alltt} % verbatim environment except that \ and braces have their usual meanings.
\usepackage{listings} % Typeset source code listings using LaTeX.
\usepackage{moreverb} % bessere verbatim-umgebungen
\usepackage{graphicx}
\usepackage{pdfpages}
\usepackage{xcolor} % Farben Definierbar \definecolor{fhorange}{RGB}{255,153,0}
\usepackage{biblatex} % F�r das Erstellen eines Literaturverzeichnisses
\addbibresource{literatur.bib}  % f�r das erstellen des Literaturverzeichnisses empfielt sich die Software Mendeley, diese erm�glicht einfaches hinzuf�gen B�chern und das exportieren in eine .bib Datei
\pagenumbering{Roman} % Nummerierung der Seiten
\newcommand{\maximagewidth}{15cm} % maximal m�gliche Bildbreite

\setlength{\parindent}{0cm} % Einr�ckung am Abstzanfang
\setlength{\parskip}{5pt plus 2pt minus 1pt} % Abstand der Abs�tze zueinander
\frenchspacing % Kein Zusatzzwischenraum nach Satzzeichen

\setcounter{secnumdepth}{3} % Z�hlung bis paragraph 1.1.1
\setcounter{tocdepth}{3} % Inhaltsverzeichnis bis paragraph 1.1.1

%\makeglossary % Schreibe ein Glossar-File

\title{Titel der Arbeit}
\author{Name des Authors}
\date{\today}

\definecolor{fhorange}{RGB}{255,153,0}
\begin{document}

%%% Titelseite %%%
\begin{titlepage}
\thispagestyle{empty}
\centerline{
\begin{minipage}{15cm}
	\flushright{
		\includegraphics[width=4cm]{fhlogo.png} \hspace{1.5cm} \ \\
		\textbf{\large Hochschule \hspace{3.5cm} ~ \\ Augsburg } \large University of \hspace{1cm} ~ \\ Applied Sciences \hspace{0.2cm} ~\\~
	}
	\flushleft{
		\textbf{\large \textcolor{fhorange}{Bachelorthesis}} \hspace{7.7cm} \textcolor{fhorange}{\large Fakultät für} \vspace{0.1cm} \\~  \hspace{10.75cm} \textcolor{fhorange}{\large Informatik}
	}
	\vspace{1cm}
	\flushleft{ \large Studienrichtung \\
		\vspace{0.2cm}
		Informatik
	}
	\vspace{0.1cm}
	\flushleft{
		\textcolor{fhorange}{\large \textbf{Nseupi Nono Hugues Landry \\
		Konfiguration und Optimierung des Embedded-Linux Betriebssystem einer Automotive Image Processing Unit}}
	}
	\vspace{0.2cm}
	\flushleft{ \large Erstprüfer: Hubert Högl \\ Zweitprüfer: \\
		\vspace{0.1cm}
		Abgabe der Arbeit am: dd.mm.yyyy
	}
	\flushleft{ In Kooperation mit Firma: EDAG Engineering GmbH\\ Hsa-Digit \\ Betreuer: Mladen Kovacev \\
		\vspace{0.5cm}
		\includegraphics[width=3cm]{hsa_digit.png} \hspace{5cm}
		\tiny{\textcolor{gray}{~\\
			\hspace{11.3cm} Hochschule für angewandte \\
			\hspace{11.3cm} Wissenschaften Augsburg \\
			\hspace{11.3cm} University of Applied Sciences \\
			\hspace{11.3cm} An der Hochschule 1 \\
			\hspace{11.3cm} D-86161 Augsburg \\
			\hspace{11.3cm} Telefon +49 821 55 86-0 \\
			\hspace{11.3cm} Fax +49 821 55 86-3222 \\
			\hspace{11.3cm} www.hs-augsburg.de \\
			\hspace{11.3cm} info@hs-augsburg.de\\
			}~\\
			\hspace{11.3cm} Fakultät für Informatik \\
			\hspace{11.3cm} Telefon +49 821 5586-3450 \\
  		\hspace{11.3cm} Nseupi Nono Hugues Landry \\
	  	\hspace{11.3cm} Salomon-Idler-str. 25 \\
			\hspace{11.3cm} 86159 Augsburg \\
			\hspace{11.3cm} Telefon +49 157 795 529 70 \\
			\hspace{11.3cm} Landrynono60@yahoo.fr \\
	}}
\end{minipage}}\par
\end{titlepage}
% Nicht über die Fehlermeldung There's no line here to end. \end... liegt daran LaTex es nicht mag wenn Leerzeilen manuell eingef�gt werden



% Zusammenfassung
\thispagestyle{empty}
\section*{Abstract}
...

\newpage
\tableofcontents
% Wenn das Inhaltsverzeichnis aktualisiert werden muss 2 mal kompilieren, erst dann wird die Aktualisierung angezeigt
\newpage
\pagenumbering{arabic}


%% Include alle Sub-Files
\include{./Einleitung/einleitung}
\include{./Grundlagen/grundlagen}
%\include{4anforderungen}
%\include{3stand}
%\include{5spezifikation}
%\include{6implementation}
%\include{7zusammenfassung}
%\include{8suffix}
% Inhaltsverzeichnis




\end{document}
